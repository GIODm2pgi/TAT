\documentclass{article}
\usepackage[utf8]{inputenc}
\usepackage[francais]{babel}
\usepackage{array}
\usepackage{graphicx}
\usepackage{fullpage}
\usepackage[svgnames]{xcolor}
\usepackage{color}
\usepackage{fourier}
\usepackage{pifont}

\newcommand*{\rotrt}[1]{\rotatebox{90}{#1}} % Command to rotate right 90 degrees
\newcommand*{\rotlft}[1]{\rotatebox{-90}{#1}} % Command to rotate left 90 degrees

\newcommand*{\titleBC}{\begingroup % Create the command for including the title page in the document
\centering % Center all text

\def\CP{\textit{\Huge Rapport Projet TAT}} % Title

\settowidth{\unitlength}{\CP} % Set the width of the curly brackets to the width of the title
\textcolor{FireBrick}{\CP} \\[\baselineskip] % Print title
{\color{Grey}\Large "Runtime Verification of User-provided Specifications"} \\ %
% Tagline or further description
\endgroup}

\definecolor{lg}{rgb}{0.9,0.9,0.9}

\let\oldv\verbatim
\let\oldendv\endverbatim

\def\verbatim{\setbox0\vbox\bgroup\oldv}
\def\endverbatim{\oldendv\egroup\fboxsep0pt \colorbox[gray]{0.9}{\usebox0}\par}

\setlength\parindent{0cm}

%%%% Le Document %%%%

\begin{document}

\begin{table}[h]
    \begin{center}
    \begin{tabular}{ >{\centering\arraybackslash}m{1.5in} >{\arraybackslash}m{4in} }

    \vspace{5mm} \includegraphics[width=2cm]{logo.png} & \vspace{9mm} Ce projet
    a été réalisé par \textbf{Johan GIRARD} et  \textbf{Pierre ODIN} dans le
    cadre du cours de TAT du Master 2 Pro. GI. 

  \end{tabular}

  \label{tabular:UKJPNdata}
  \end{center}
\end{table}

\hrule\hrule

\vspace{1.5cm}

\titleBC

\vspace{1cm}

\section{Iterator}

\subsection{Monitor et AspectJ}

Cette partie du projet reprend la base fournie. L'automate implémenté est le
suivant :

\begin{center}
\includegraphics[width=10cm, clip, trim=5.5cm 18cm 5.5cm
4cm]{iterator.pdf}
\end{center}

Le moniteur passe dans l'état \texttt{Next} après un appel à \texttt{hasNext()}
et dans l'état \texttt{HasNext} après un appel à \texttt{next()}. Un appel à
\texttt{next()} est autorisé que dans l'état \texttt{Next}. Les classes
\texttt{AspectJ} utilisée sont celles fournies.

\subsection{Tests}

Les tests sont les suivants :

\begin{itemize}
  \item \texttt{Iterator\_OK.java} : parcours de deux \texttt{Vector} avec appel
  à \texttt{next()} dans une boucle qui test \texttt{hasNext()}.
  \item \texttt{Iterator\_ERROR\_DOUBLE\_NEXT.java} : parcours de deux
  \texttt{Vector} avec deux appels à \texttt{next()} dans une boucle qui test
  \texttt{hasNext()}. Le deuxième appel provoque une erreur.
\end{itemize}

\subsection{Overhead}

La classe \texttt{Iterator\_INTENSE\_USE.java} permet de mesurer l'overhead en
temps.
La fonction principale parcourt un grand nombre d'\texttt{Iterator}. Les paramètres
et les résultats de cette mesure sont les suivants :

\vspace{0.3cm}
\begin{verbatim}
     /**
     * Aj : AjParametricHasNext.aj
     * Config : no print in monitors 
     * Number of exec : 100
     * 
     *                      avg execution time with monitor      (137.14)
     * Overhead in time = ------------------------------------ = ------- = 8.29
     *                     avg execution time without monitor    (16.54)
     */
\end{verbatim}

\section{Enumeration}

\subsection{Monitor}

L'automate implémenté est le
suivant :

\begin{center}
\includegraphics[width=14.5cm, clip, trim=3cm 12.5cm 0cm
4cm]{enumeration.pdf}
\end{center}

L'évènement \texttt{update} permet de mettre à jour l'état d'un \texttt{Vector}.
L'évènement \texttt{init} permet de stocker l'état d'un \texttt{Vector} au
moment de son assignation dans une \texttt{Enumeration}. À chaque appel à
\texttt{nextElement()}, le moniteur vérifie que les états sont égaux. Dans notre
implémentation, une état correspond au nombre de modification
(\texttt{add(\ldots) or \texttt{remove(\ldots)}}) du \texttt{Vector}. Lors de
l'initialisation, l'état vaut \texttt{0} et à chaque mise à jour, l'état prend
la valeur \texttt{state + 1}.

\paragraph{\danger} Un ajout d'un élément suivie d'une suppression immédiate de
cet élément entre deux appels à \texttt{nextElement()} provoque une erreur même
si le contenu du \texttt{Vector} est resté identique.

\subsection{AspectJ}

Un appel à \texttt{Vector.add(\ldots)} ou \texttt{Vector.remove(\ldots)} propage
un évènement \texttt{update}. De plus, un appel à
\texttt{Enumeration.create(\ldots)} propage un évènement \texttt{init}. Un appel à \texttt{Enumeration.nextElement()}
propage un évènement \texttt{nextElement}.

\vspace{0.3cm}

La classe \texttt{Enumeration.java} est une classe que nous avons implémentée
pour faciliter la création des \texttt{pointcut}. En effet, nous n'arrivions pas
à intercepter la création de l'\texttt{Enumeration} en \texttt{AspectJ}, donc
nous avons créer une classe intermédiaire qui nous permet d'ajouter une fonction
\texttt{create(\ldots)} qui est facilement interceptable.

\subsection{Tests}

Les tests sont les suivants :

\begin{itemize}
  \item \texttt{Enumeration\_OK.java} : parcours de deux \texttt{Enumeration} sans modification des \texttt{Vector}.
  \item \texttt{Enumeration\_ADD.java} : parcours de deux \texttt{Enumeration}
  avec modification des \texttt{Vector} après le parcours.
  \item \texttt{Enumeration\_RECREATE.java} : parcours de deux \texttt{Enumeration}
  avec modification des \texttt{Vector} puis re-création et un nouveau
  parcours.
  \item \texttt{Enumeration\_ERROR\_ADD.java} : parcours de deux \texttt{Enumeration}
  avec modification des \texttt{Vector} pendant le parcours.
  \item \texttt{Enumeration\_ERROR\_ADD\_REMOVE.java} : parcours de deux \texttt{Enumeration}
  avec modifications des \texttt{Vector} (add puis remove) pendant le parcours.
  Ce test montre que des modifications qui ne change pas l'état de l'objet
  provoque quand même une erreur.
  \item \texttt{Enumeration\_ERROR\_LAST.java} : parcours de deux \texttt{Enumeration}
  avec modification des \texttt{Vector} (ajout) pendant le parcours à la
  dernière itération.
\end{itemize}

\subsection{Overhead}

La classe \texttt{Enumeration\_INTENSE\_USE.java} permet de mesurer l'overhead
en temps.
La fonction principale exécute un grand nombre de fois les tests donnant un
résultat valide.

\vspace{0.3cm}
\begin{verbatim}
     /**
     * Aj : AjEnumeration.aj
     * Config : no print in monitors 
     * Number of exec : 100
     * 
     *                       avg execution time with monitor     (123.72)
     * Overhead in time = ------------------------------------ = ------- = 41.94
     *                     avg execution time without monitor     (2.95)
     */
\end{verbatim}

\section{Hashcode}

\subsection{Monitor}

\begin{center}
\includegraphics[width=16cm, clip, trim=3.5cm 11.5cm 0cm
6.33cm]{hashcode.pdf}
\end{center}

\subsection{AspectJ}

\subsection{Tests}

\subsection{Overhead}

\section{File (Q3)}

\subsection{Monitor}

\begin{center}
\includegraphics[width=16cm, clip, trim=3.5cm 14.5cm 3.5cm
4.5cm]{q3.pdf}
\end{center}

\subsection{AspectJ}

\subsection{Tests}

\subsection{Overhead}

\newpage

\paragraph{Description de l'invariant :} Il doit y avoir au maximum 50 incompatibilités renseignées dans le tableau \texttt{incomp}. La valeur de la variable \texttt{nb\_inc} doit être positive et doit être inférieur à 50 car ce nombre est utilisé pour gérer les indices du tableau \texttt{incomp}, il doit donc être situé entre l'indice minimum (0) et l'indice maximum (49).

\vspace{-0.2cm}
\paragraph{Test invalidant l'invariant :} Il s'agit de \texttt{testFailInvariant1()} dans \texttt{TestExplosivesJUnit4.java}.

\vspace{-0.2cm}
\paragraph{Description du test :} Le test fait trop d'appels à \texttt{add\_incomp(...)} et \texttt{nb\_inc} devient supérieur à 49.  

\subsection{Invariant \no2}

\paragraph{Description de l'invariant :} Il doit y avoir au maximum 30 affectations renseignées dans le tableau \texttt{assign}. La valeur de la variable \texttt{nb\_assign} doit être positive et doit être inférieur à 30 car ce nombre est utilisé pour gérer les indices du tableau \texttt{assign}, il doit donc être situé entre l'indice minimum (0) et l'indice maximum (29). Cet invariant indique également qu'il peut y avoir au maximum 30 bâtiments différents répertoriés dans \texttt{assign}.

\vspace{-0.2cm}
\paragraph{Test invalidant l'invariant :} Il s'agit de \texttt{testFailInvariant2()} dans \texttt{TestExplosivesJUnit4.java}.

\vspace{-0.2cm}
\paragraph{Description du test :} Le test fait trop d'appels à \texttt{add\_assign(...)} et \texttt{nb\_assign} devient supérieur à 29. 

\subsection{Invariant \no3}

\paragraph{Description de l'invariant :} Tous les produits doivent être référencés par un nom ayant pour préfixe \texttt{"Prod"} dans la liste des incompatibilités ($\Leftrightarrow$ toutes les valeurs définies dans le tableau \texttt{incomp} doivent commencer par \texttt{"Prod"}).

\vspace{-0.2cm}
\paragraph{Test invalidant l'invariant :} Il s'agit de \texttt{testFailInvariant3()} dans \texttt{TestExplosivesJUnit4.java}.

\vspace{-0.2cm}
\paragraph{Description du test :} Le test fait un ajout d'une incompatibilité dont le nom des produits ne commence pas par \texttt{"Prod"} (un seul nom de produit ne commençant pas par \texttt{"Prod"} serait suffisant pour invalider l'invariant).

\subsection{Invariant \no4}

\paragraph{Description de l'invariant :} Tous les bâtiments doivent être référencés par un nom ayant pour préfixe \texttt{"Bat"} et tous les produits doivent être référencés par un nom ayant pour préfixe \texttt{"Prod"} dans la liste des affectations. Cela signifie que toutes les valeurs définies dans le tableau \texttt{assign} doivent commencer par \texttt{"Bat"} pour le premier élément et par \texttt{"Prod"} sur le second.

\vspace{-0.2cm}
\paragraph{Test invalidant l'invariant :} Il s'agit de \texttt{testFailInvariant4()} dans \texttt{TestExplosivesJUnit4.java}.

\vspace{-0.2cm}
\paragraph{Description du test :} Le test fait un ajout d'une assignation dont le nom du bâtiment ne commence pas par \texttt{"Bat"} et le nom du produit ne commence pas par \texttt{"Prod"} (une seule de ces deux irrégularités serait suffisante pour invalider l'invariant).

\subsection{Invariant \no5}

\paragraph{Description de l'invariant :} Un produit ne doit pas être incompatible avec lui-même. Cela signifie que le tableau \texttt{incomp} ne doit pas avoir une entrée avec deux fois le même produit.

\vspace{-0.2cm}
\paragraph{Test invalidant l'invariant :} Il s'agit de \texttt{testFailInvariant5()} dans \texttt{TestExplosivesJUnit4.java}.

\vspace{-0.2cm}
\paragraph{Description du test :} Le test fait un ajout d'une incompatibilité où la paire de produits contient deux fois le même produit.

\subsection{Invariant \no6}

\paragraph{Description de l'invariant :} Si un produit \texttt{X} est incompatible avec un produit \texttt{Y}, alors le produit \texttt{Y} est incompatible avec le produit \texttt{X}. Cela signifie que le tableau \texttt{incomp} doit contenir deux entrées pour chaque incompatibilité : la première de la forme \texttt{[X][Y]} et l'autre de la forme \texttt{[Y][X]}.

\vspace{-0.2cm}
\paragraph{Test invalidant l'invariant :} Il s'agit de \texttt{testFailInvariant6()} dans \texttt{TestExplosivesJUnit4Public.java}.

\vspace{-0.2cm}
\paragraph{Description du test :} La fonction \texttt{add\_incomp(...)} ajoutent deux entrées dans le tableau \texttt{incomp} (\texttt{[X][Y]} et \texttt{[Y][X]}) à chaque appel. Il n'est donc pas possible d'invalider cette propriété en utilisant uniquement les méthodes de la classe. Le test modifie donc un attribut public de la classe pour invalider l'invariant : une incompatibilité est ajoutée, puis la deuxième entrée du tableau est retirée et enfin un appel à \texttt{skip()} est fait.

\subsection{Invariant \no7}

\paragraph{Description de l'invariant :} Deux produits assignés au même bâtiment ne doivent pas être incompatibles. Cela signifie que si le tableau \texttt{assign} contient une entrée \texttt{[B][P1]} et une entrée \texttt{[B][P2]}, alors les entrées \texttt{[P1][P2]} et \texttt{[P2][P1]} ne doivent pas se trouver dans le tableau \texttt{incomp}.

\vspace{-0.2cm}
\paragraph{Test invalidant l'invariant :} Il s'agit de \texttt{testFailInvariant7()} dans \texttt{TestExplosivesJUnit4.java}.

\vspace{-0.2cm}
\paragraph{Description du test :} Le test ajoute des incompatibilités entre des produits, puis il ajoute deux assignations dans le même bâtiment de deux produits déclarés incompatibles.

\newpage

\section{Calcul de préconditions}

\subsection{Méthode add\_incomp(...)}

\noindent
La précondition réalisée pour la méthode \texttt{add\_incomp(...)} est la suivante :
\vspace{0.3cm}

\noindent
\begin{verbatim}
    /*@ requires (0 <= nb_inc+2 && nb_inc+2 < 50) ;
      @ requires (prod1.startsWith("Prod") && prod2.startsWith("Prod")) ; 
      @ requires (!prod1.equals(prod2)) ;
      @*/
\end{verbatim}

\noindent
Cette précondition vérifie trois éléments :
\begin{itemize}
\item La valeur de \texttt{nb\_inc} en entrée puis modifiée (ajout de 2 comme dans l'implémentation de la méthode) est comprise en 0 et 49.
\item Les arguments de la méthode sont préfixés par \texttt{"Prod"}.
\item Les deux arguments ne sont pas égaux.
\end{itemize}

\vspace{0.3cm}
\noindent
En ajoutant cette précondition, les tests suivants deviennent inconclusifs :
\begin{itemize}
\renewcommand{\labelitemi}{$\rightarrow$} 
\item \texttt{testFailInvariant1()} ; \texttt{testFailInvariant3()} ; \texttt{testFailInvariant5()}
\end{itemize}

\subsection{Méthode add\_assign(...)}

\noindent
La précondition réalisée pour la méthode \texttt{add\_assign(...)} est la suivante :
\vspace{0.3cm}

\noindent
\begin{verbatim}
    /*@ requires (0 <= nb_assign+1 && nb_assign+1 < 30) ;
      @ requires (bat.startsWith("Bat") && prod.startsWith("Prod")) ;
      @ requires (\forall int i; 0 <= i &&  i < nb_assign;
      @	            (\forall int j; 0 <= j && j < nb_inc;
      @	               ( (assign[i][0].equals(bat) && incomp[j][0].equals(assign[i][1]))
      @	               ==> ( !incomp[j][1].equals(prod) )
      @          ))) ;
      @*/
\end{verbatim}
\vspace{0.2cm}

\noindent
Cette précondition vérifie trois éléments :
\begin{itemize}
\item La valeur de \texttt{nb\_assign} en entrée puis modifiée (ajout de 1 comme dans l'implémentation de la méthode) est comprise en 0 et 29.
\item Le premier argument de la méthode est préfixé par \texttt{"Bat"} et le second par \texttt{"Prod"}.
\item Il n'y a pas de produits incompatibles avec le produit passé en argument dans le bâtiment passé en argument. Cela se traduit par une précondition qui vérifie que s'il existe un produit incompatible avec un produit du bâtiment passé en argument, alors ce produit n'est pas celui passé en argument.
\end{itemize}

\vspace{0.3cm}
\noindent
En ajoutant cette précondition, les tests suivants deviennent inconclusifs :
\begin{itemize}
\renewcommand{\labelitemi}{$\rightarrow$} 
\item \texttt{testFailInvariant2()} ; \texttt{testFailInvariant4()} ; \texttt{testFailInvariant7()}
\end{itemize}

\section{Recherche d'un bâtiment}

\paragraph{Précision sur l'énoncé :} La méthode \texttt{findBat(...)} retourne un nouveau nom de bâtiment dans le cas où aucun des bâtiments référencés dans le tableau \texttt{assign} ne peut recevoir le nouveau produit. La solution \textit{"trop simple"} qui consiste à retourner un nouveau bâtiment à chaque appel n'est donc pas autorisée avec cet énoncé. Cependant, nous fournissons tout de même des méthodes et un fichier de test permettant de mettre en évidence les différences entre la solution \textit{"trop simple"} et notre solution pour \texttt{findBat(...)}.

\subsection{Méthode compatible(...) et existe\_bat(...)}

\noindent
\begin{verbatim}
    /*@ requires (prod1.startsWith("Prod") && prod2.startsWith("Prod")) ;
      @ ensures (\result == true) ==>
      @     (\forall int i; 0 <= i && i < nb_inc;
      @	        !(incomp[i][0].equals(prod1) && incomp[i][1].equals(prod2)) ) ;
      @ ensures (\result == false) ==>
      @	    (\exists int i; 0 <= i && i < nb_inc; 
      @	        incomp[i][0].equals(prod1) && incomp[i][1].equals(prod2) ) ;
      @*/
    public /*@ pure @*/ boolean compatible (String prod1, String prod2){ ... }
\end{verbatim}
\vspace{0.2cm}

\noindent
La méthode \texttt{compatible(...)} teste si deux produits sont déclarés comme compatibles. Sa spécification vérifie d'une part que les arguments sont valides et d'autre part que la méthode retourne \texttt{true} que si les produits sont compatibles et \texttt{false} que s'ils ne le sont pas.

\vspace{0.3cm}
\noindent
\begin{verbatim}
    /*@ requires (bat.startsWith("Bat")) ;
      @ ensures (\result == true) ==>
      @	    (\forall int i; 0 <= i &&  i < nb_assign;
      @	        !assign[i][0].equals(bat)) ;
      @ ensures (\result == false) ==>
      @     (\exists int i; 0 <= i && i < nb_assign;
      @	        assign[i][0].equals(bat)); 
      @*/
    public /*@ pure @*/ boolean existe_bat (String bat){ ... }
\end{verbatim}
\vspace{0.2cm}

\noindent
La méthode \texttt{existe\_bat(...)} teste si un bâtiment stocke au moins un produit. Sa spécification vérifie d'une part que l'argument est valide et d'autre part que la méthode retourne \texttt{true} que si le bâtiment n'apparaît pas dans \texttt{assign} et \texttt{false} que s'il y a au moins une entrée dans \texttt{assign} pour le bâtiment.

\subsection{Méthode findBat(...)}

\noindent
\begin{verbatim}
    /*@ requires (prod.startsWith("Prod")) ;
      @ ensures (\result.startsWith("Bat")) ;
      @ ensures (\forall int i; 0 <= i &&  i < nb_assign;
      @	    (assign[i][0].equals(\result)) ==> (compatible(assign[i][1],prod)) );
      @ ensures (existe_bat(\result)) ==>
      @	    (\forall int i; 0 <= i && i < nb_assign;
      @	      (\exists int k; 0 <= k && k < nb_assign;
      @	        (assign[i][0].equals(assign[k][0]) && !compatible(assign[k][1],prod))));
      @*/
\end{verbatim}
\vspace{0.2cm}

\noindent
Cette spécification vérifie les éléments suivants :
\begin{itemize}
\item L'argument et le résultat sont syntaxiquement valides.
\item Le produit est stocké dans un bâtiment où il est compatible avec tous les autres produits stockés dans ce bâtiment.
\item Si le bâtiment retourné ne stockait aucun produit avant l'appel à \texttt{findBat(...)}, alors cela signifie que le produit ne peut être stocké dans aucun des bâtiments déclarés dans \texttt{assign}. Cette post-condition est utilisé pour interdire la solution \textit{"trop simple"}.
\end{itemize}

\newpage

\noindent
Le principe de la méthode \texttt{findBat(...)} est le suivant :

\vspace{0.2cm}
\noindent
\begin{verbatim}
    public String findBat (String  prod){
        // On crée une liste de tous les bâtiments (L1)
        // On crée une liste de bâtiments qui ne peuvent pas recevoir le produit (L2)
        // On parcourt le tableau "assign" pour remplir ces deux listes
                // On test la compatibilité
                // Si le bâtiment n'est pas dans (L1), on l'ajoute dans (L1)
                // Si le bâtiment contient un produit incompatible, on l'ajoute dans (L2)
        // On cherche le premier bâtiment de (L1) qui n'est pas dans (L2)
        // S'il n'y en a pas, on retourne un nouveau bâtiment
    }
\end{verbatim}
\vspace{0.2cm}

\noindent
La méthode est testée dans le fichier \texttt{TestExplosivesJUnit4FindBat.java}. Le test le plus intéressant est implémenté dans le fichier \texttt{testFindBat7()}. Il constitue une liste de 17 produits, il crée ensuite plusieurs incompatibilités entre certains de ces produits et enfin il cherche un bâtiment pour chaque produit. La trace des assignations permet de vérifier que les incompatibilités ne sont pas violées.

\paragraph{\danger} L'implémentation de \texttt{findBat(...)} retourne le premier bâtiment qui peut stocker le produit. La répartition proposée après un test de ce type est donc valide mais elle n'est pas forcément optimale (cela dépend de l'ordre d'insertion dans le tableau \texttt{assign}). Une amélioration possible pourrait être de ré-organiser les assignations dans \texttt{findBat(...)} pour toujours assurer que la répartition utilise le moins de bâtiments possibles.


\subsection{Comparaison des méthodes findBatSimple(...), findBatSimpleInterdit(...) et findBat(...)}

\noindent
Le fichier de \texttt{TestExplosivesJUnit4FindBatVSFindBatSimple.java} permet de mettre en évidence le rôle de la post-condition que nous avons ajoutée pour éviter la solution \textit{"trop simple"}. Le test réalisé est celui décrit dans la section précédente avec à chaque fois une méthode différente. Le tableau suivant résume cette comparaison :

\begin{table}[h]
    \begin{center}
    \begin{tabular}{ | >{\centering\arraybackslash}m{5cm} | >{\centering\arraybackslash}m{3cm} | >{\centering\arraybackslash}m{2.7cm} | >{\centering\arraybackslash}m{4cm} |}
    \hline

\textbf{Méthode utilisée} & \textbf{Post-condition anti-\textit{"trop simple"}} & \textbf{Implémentation \textit{simple}} & \textbf{Résultat du test} \\ \hline
\texttt{findBatSimple(...)} & \ding{55} & \ding{51} & \texttt{Validé} :

1 produit = 1 bâtiment \\ \hline
\texttt{findBatSimpleInterdit(...)} & \ding{51} & \ding{51} & \vspace{1mm} \texttt{FAILURE} \vspace{1mm} \\ \hline
\texttt{findBat(...)} & \ding{51} & \ding{55} & \texttt{Validé} :

17 produits $\rightarrow$ 3 bâtiments \\ \hline
  \end{tabular}
  \end{center}
\end{table}

\end{document}